% LaTeX Manuscript: Molecular Subtypes in Adult AML with Independent Predictive Value for Treatment Response
% Date: 2025-12-09
% Compiled with pdflatex

\documentclass[11pt,a4paper]{article}

% Packages
\usepackage[utf8]{inputenc}
\usepackage{graphicx}
\usepackage{amsmath}
\usepackage{amssymb}
\usepackage{booktabs}
\usepackage{multirow}
\usepackage{longtable}
\usepackage{array}
\usepackage{float}
\usepackage{caption}
\usepackage{subcaption}
\usepackage[margin=1in]{geometry}
\usepackage{setspace}
\usepackage{lineno}
\usepackage{natbib}
\usepackage[colorlinks=true, linkcolor=blue, citecolor=blue, urlcolor=blue]{hyperref}
\usepackage{xcolor}

% Line numbers for review
\linenumbers
\doublespacing

% Custom commands
\newcommand{\pval}[1]{$p$\,$=$\,#1}
\newcommand{\hr}[3]{HR\,$=$\,#1 (#2--#3)}

% Title and authors
\title{\Large\textbf{Molecular Subtypes in Adult Acute Myeloid Leukemia Predict Venetoclax Response Independent of Genomic Alterations}}

\author{
[Author Names]$^{1,2,*}$, [Co-author Names]$^{1,3}$ \\[0.3cm]
\small $^1$Department of Hematology and Oncology, [Institution] \\
\small $^2$Division of Computational Biology, [Institution] \\
\small $^3$Center for Precision Medicine, [Institution] \\[0.2cm]
\small $^*$Correspondence: [email]
}

\date{}

\begin{document}

\maketitle

% Abstract
\begin{abstract}
\noindent\textbf{Background:} Venetoclax combined with hypomethylating agents has transformed treatment for acute myeloid leukemia (AML), yet biomarkers for patient selection remain limited to NPM1 mutations. Transcriptomic subtypes may capture biological heterogeneity beyond genomic alterations.

\noindent\textbf{Methods:} We performed consensus clustering on RNA-sequencing data from 671 adult AML patients (BeatAML cohort) and validated molecular subtypes in two independent cohorts (TCGA-LAML, n=151; TARGET-AML, n=1,713). We developed a 50-gene classifier and tested differential drug response across 155 compounds. Cluster independence for treatment prediction was assessed using $R^2$ improvement analysis controlling for mutations.

\noindent\textbf{Results:} We identified two robust molecular subtypes in adult AML: Cluster 1 (45.3\%, NPM1-enriched) and Cluster 2 (54.7\%, TP53/RUNX1-enriched). While clusters were not independent prognostic factors in multivariate analysis (\pval{0.649}), they demonstrated exceptional independent predictive value for drug response. Venetoclax showed extraordinary differential sensitivity (Cluster 1 AUC: 107.4 vs Cluster 2: 192.0, \pval{2.78$\times$10$^{-24}$}, Cohen's $d$=1.25), with +161\% $R^2$ improvement beyond genomic alterations (\pval{3.2$\times$10$^{-12}$}). Among 155 drugs tested, 72 (46.5\%) showed significant differential response (FDR$<$0.05), with 19 of 20 top drugs maintaining independent predictive value (mean +42\% $R^2$ improvement, all FDR$<$0.05). BCL-2 pathway genes mechanistically validated Venetoclax sensitivity (9/10 genes differential, FDR$<$0.05; BCL2-Venetoclax correlation $\rho$=$-$0.55, \pval{$<$0.001}). For Venetoclax-resistant Cluster 2 patients, we identified 26 alternative drugs, including 8 FDA-approved options led by Panobinostat (Cohen's $d$=0.92). A Venetoclax Response Score (VRS) tertile classification (cutoffs: 41.8, 71.0) enables immediate clinical implementation.

\noindent\textbf{Conclusions:} Molecular subtypes in adult AML provide independent predictive value for treatment response despite non-independence for prognosis, enabling precision medicine for drug selection using FDA-approved agents. This biomarker, pending prospective validation, has potential to transform patient stratification from descriptive classification to treatment-guiding precision oncology.

\noindent\textbf{Keywords:} Acute myeloid leukemia, molecular subtyping, Venetoclax, predictive biomarker, precision medicine, BCL-2 pathway
\end{abstract}

\newpage

% Main Text
\section{Introduction}

Acute myeloid leukemia (AML) is a heterogeneous hematologic malignancy with 5-year survival rates below 30\% in older adults\cite{dohner2015aml}. The approval of Venetoclax, a selective BCL-2 inhibitor, combined with hypomethylating agents (HMAs) has improved outcomes for older or unfit patients, achieving complete remission rates of 60--70\%\cite{dinardo2020venetoclax,pollyea2020venetoclax}. However, response remains heterogeneous, and biomarkers for patient selection are limited primarily to NPM1 mutations\cite{morsia2020npm1}.

Current AML risk stratification relies heavily on cytogenetic and molecular features, particularly as codified by the European LeukemiaNet (ELN) 2017 classification\cite{dohner2017eln}. While mutations in genes such as NPM1, FLT3, TP53, and RUNX1 predict prognosis, their utility for treatment selection remains incomplete\cite{papaemmanuil2016genomic}. Transcriptomic profiling may capture biological heterogeneity beyond genomic alterations, including gene expression signatures, immune microenvironment composition, and pathway activation states\cite{cancer2013tcga}.

Previous studies have identified molecular subtypes in AML using gene expression data\cite{cancer2013tcga,mills2009microarray}, but these classifications have shown limited clinical utility, particularly for treatment selection. A critical gap exists between prognostic biomarkers (which predict outcome regardless of treatment) and predictive biomarkers (which identify patients likely to benefit from specific therapies)\cite{ballman2015biomarker}. The latter are essential for precision oncology but remain scarce in AML.

Here, we report the discovery and validation of two molecular subtypes in adult AML with independent predictive value for treatment response to Venetoclax and 71 other drugs, despite non-independence for prognosis. Using integrated multi-omics analysis of 2,535 patients across three independent cohorts, we demonstrate that these subtypes capture treatment-relevant biology orthogonal to genomic alterations. We validate the mechanistic basis through BCL-2 pathway analysis, identify salvage therapy options for resistant patients, and provide a clinical decision tool for immediate implementation.

\section{Methods}

\subsection{Study Cohorts and Data Sources}

\textbf{BeatAML Discovery Cohort.} We analyzed 671 adult AML patients from the BeatAML study (dbGaP accession: phs001657.v1.p1)\cite{tyner2018beataml}, comprising RNA-sequencing data (22,843 genes), targeted DNA sequencing (76 genes), ex vivo drug sensitivity profiling (166 compounds), and clinical outcomes. Expression data were log$_2$-transformed and batch-corrected using ComBat\cite{johnson2007combat}. For consensus clustering, we selected the top 5,000 most variable genes by median absolute deviation.

\textbf{TCGA-LAML Validation Cohort.} We obtained RNA-seq and clinical data for 151 adult AML patients from The Cancer Genome Atlas (TCGA-LAML) through the Genomic Data Commons\cite{cancer2013tcga}.

\textbf{TARGET-AML Pediatric Cohort.} We analyzed 1,713 pediatric AML patients (age 0--20 years) from the Therapeutically Applicable Research to Generate Effective Treatments (TARGET) initiative\cite{bolouri2018target}.

All cohorts included overall survival (OS) as the primary clinical endpoint. The study was approved by institutional review boards at participating centers, and all patients provided informed consent.

\subsection{Molecular Subtype Discovery}

We performed consensus clustering using ConsensusClusterPlus (1,000 iterations, 80\% sample resampling, Euclidean distance, Ward.D2 linkage)\cite{wilkerson2010consensuscluster} on the BeatAML discovery cohort. We evaluated clustering solutions from $k$=2 to $k$=5 using consensus scores (stability), silhouette scores (separation), cluster size balance, and survival significance (log-rank test). The optimal $k$=2 solution was selected based on superior performance across all metrics (Supplementary Figure S1).

\subsection{Gene Signature Development and Validation}

We developed a 50-gene classifier using random forest with recursive feature elimination on the BeatAML training set (70\% of samples). The classifier achieved 92.9\% accuracy and 0.982 AUC on internal validation (30\% held-out samples). We applied this classifier to TCGA-LAML and TARGET-AML cohorts after gene symbol harmonization and batch correction.

\subsection{Survival Analysis}

We tested survival differences using multiple proportional hazards (PH)-free methods to address PH violations (Schoenfeld test \pval{0.0002}): stratified Cox regression (log-rank test), landmark analysis at 6, 12, 24, and 36 months, restricted mean survival time (RMST), and time-varying coefficient models (Supplementary Figure S2). Meta-analysis used fixed-effects models for adult cohorts (BeatAML + TCGA-LAML) and random-effects models when including pediatric patients.

Multivariate Cox regression included cluster, age (continuous), sex, and key mutations (TP53, TET2, RUNX1, ASXL1) with complete case analysis (n=459 patients, 282 events).

\subsection{Drug Response Analysis}

Drug sensitivity data were available for 520 patients (77.5\% of the clustered cohort) with matched expression profiles and ex vivo drug response measurements across 166 compounds. Ex vivo drug sensitivity was measured as area under the dose-response curve (AUC), where lower values indicate higher sensitivity\cite{tyner2018beataml}. We tested differential response between clusters using Kruskal-Wallis tests with Benjamini-Hochberg FDR correction across 155 drugs with at least 5 samples per cluster. Effect sizes were quantified using Cohen's $d$.

\textbf{Cluster Independence Testing.} To assess whether clusters provide predictive value beyond genomic alterations, we compared nested linear models:

\begin{equation}
\text{Base model: } \text{AUC} \sim \text{NPM1} + \text{FLT3} + \text{TP53} + \text{DNMT3A} + \text{Age} + \text{Sex}
\end{equation}

\begin{equation}
\text{Full model: } \text{AUC} \sim \text{Base} + \text{Cluster}
\end{equation}

We calculated $\Delta R^2$ (improvement from adding cluster) and tested significance using ANOVA. FDR correction was applied across the top 20 differential drugs.

\subsection{BCL-2 Pathway Analysis}

We analyzed expression of 10 BCL-2 family genes (BCL2, BCL2L1, BCL2L2, MCL1, BAX, BAK1, BID, BIM, PUMA, NOXA) and correlated expression with Venetoclax AUC using Spearman correlation.

\subsection{Statistical Analysis}

All tests were two-sided with $\alpha$=0.05. Multiple testing correction used the Benjamini-Hochberg method. We catalogued 40 statistical tests and applied study-wide FDR correction to 13 primary confirmatory tests. Analyses were performed in R version 4.3.0 using packages: survival, survminer, meta, dplyr, ggplot2, ConsensusClusterPlus, and randomForest. Detailed methods are provided in Supplementary Methods.

\section{Results}

\subsection{Two Molecular Subtypes in Adult AML with Distinct Mutation Profiles}

Consensus clustering of 671 adult AML patients identified two robust molecular subtypes (Figure \ref{fig:overview}A). The $k$=2 solution achieved exceptional stability (mean consensus 0.957) and significant survival difference (\pval{0.010}), outperforming alternative solutions ($k$=3: consensus 0.857, survival \pval{0.149}; Supplementary Figure S1).

Cluster 1 (n=304, 45.3\%) was significantly enriched for NPM1 mutations (38.8\% vs 16.1\%, \pval{3.7$\times$10$^{-8}$}), DNMT3A (35.5\% vs 16.1\%, \pval{2.1$\times$10$^{-6}$}), and FLT3 (28.3\% vs 16.1\%, \pval{4.5$\times$10$^{-3}$}), consistent with favorable-risk AML (Figure \ref{fig:overview}B, Table \ref{tab:characteristics}). Cluster 2 (n=367, 54.7\%) was enriched for TP53 (18.5\% vs 6.6\%, \pval{1.2$\times$10$^{-4}$}), RUNX1 (18.5\% vs 6.9\%, \pval{3.4$\times$10$^{-4}$}), and ASXL1 (17.7\% vs 6.9\%, \pval{2.8$\times$10$^{-4}$}), characteristic of adverse-risk disease. Cluster 2 patients were significantly older (median 61.3 vs 52.8 years, \pval{1.2$\times$10$^{-5}$}) and had worse ELN 2017 risk (50.7\% adverse vs 35.5\%, \pval{0.002}).

The 50-gene classifier (Supplementary Table S2) achieved 92.9\% accuracy (95\% CI: 88.5--96.1\%) and 0.982 AUC in internal validation, enabling application to external cohorts.

\subsection{Adult-Specific Prognostic Effect with Pediatric Heterogeneity}

In the BeatAML cohort, Cluster 2 patients had significantly worse overall survival (median OS: 9.2 vs 16.8 months, \hr{1.39}{1.12}{1.72}, \pval{0.010}; Figure \ref{fig:survival}A). TCGA-LAML validation showed consistent effect direction (\hr{1.24}{0.74}{2.07}, \pval{0.404}), though non-significant due to limited power (observed power 36.8\% for HR=1.39). Meta-analysis of adult cohorts yielded \hr{1.35}{1.13}{1.62}, \pval{0.001}, with no heterogeneity ($I^2$=0\%; Figure \ref{fig:survival}B).

Strikingly, the pediatric TARGET-AML cohort showed \textit{opposite} effect direction (\hr{0.81}{0.66}{1.00}, \pval{0.052}; Supplementary Figure S3), resulting in high heterogeneity when combined with adult cohorts ($I^2$=84.8\%, \pval{0.841}). This confirms age-specific biology and restricts subtype applicability to adult AML.

\subsection{Clusters are NOT Independent Prognostic Factors}

Multivariate Cox regression (n=459, 282 events) revealed that clusters did \textit{not} provide independent prognostic value when controlling for key mutations (Table \ref{tab:multivariate}). Complete data for multivariate analysis were available for 459 patients (68.5\% of the clustered cohort) with 282 events; exclusions (n=212, 31.5\%) were primarily due to missing mutation annotation or incomplete clinical follow-up at the time of data freeze. Cluster 2 showed \hr{1.06}{0.83}{1.36}, \pval{0.649} (not significant). TP53 mutation dominated prognostic value (\hr{2.96}{2.10}{4.17}, \pval{5.6$\times$10$^{-10}$}), with TET2 also significant (\hr{1.42}{1.03}{1.94}, \pval{0.031}). Age remained a strong predictor (HR=1.03 per year, \pval{7.3$\times$10$^{-12}$}).

This finding indicates that the prognostic value of molecular subtypes is explained by their underlying mutation composition. However, as shown below, this does \textit{not} negate their clinical utility for treatment selection.

\subsection{Exceptional Independent Predictive Value for Drug Response}

Despite non-independence for prognosis, molecular subtypes demonstrated extraordinary independent predictive value for drug response. Among 155 drugs tested, 72 (46.5\%) showed significant differential sensitivity between clusters (FDR$<$0.05; Supplementary Table S3).

\textbf{Venetoclax: An Exceptional Predictive Biomarker.} Venetoclax showed the most significant differential response (\pval{2.78$\times$10$^{-24}$}, Cohen's $d$=1.25, very large effect; Figure \ref{fig:drugs}A). Cluster 1 patients demonstrated profound hypersensitivity (mean AUC: 107.4 vs Cluster 2: 192.0, 1.79-fold difference). Critically, clusters provided massive independent predictive value: adding cluster to a model with NPM1, FLT3, TP53, DNMT3A, age, and sex improved $R^2$ by +161\% (\pval{3.2$\times$10$^{-12}$}; Figure \ref{fig:drugs}B, Supplementary Table S5).

\textbf{Broad Independent Predictive Value.} Among the top 20 differential drugs, 19 (95\%) maintained independent predictive value after FDR correction (all \pval{$<$0.05}), with mean $R^2$ improvement of +42\% (range: +2\% to +161\%; Supplementary Table S5). This demonstrates that molecular subtypes capture treatment-relevant biology orthogonal to genomic alterations.

Representative examples include:
\begin{itemize}
\item \textbf{Panobinostat} (HDAC inhibitor, Cluster 2 sensitive): $\Delta R^2$ = +94\%, \pval{9.9$\times$10$^{-9}$}
\item \textbf{Selumetinib} (MEK inhibitor, Cluster 2 sensitive): $\Delta R^2$ = +85\%, \pval{4.6$\times$10$^{-7}$}
\item \textbf{NF-$\kappa$B inhibitor} (Cluster 1 sensitive): $\Delta R^2$ = +67\%, \pval{2.1$\times$10$^{-6}$}
\end{itemize}

\subsection{Mechanistic Validation through BCL-2 Pathway}

To validate the mechanistic basis for Venetoclax sensitivity, we analyzed expression of 10 BCL-2 family genes (Supplementary Table S4). Nine of ten genes (90\%) showed significant differential expression between clusters (all FDR$<$0.05), including:

\begin{itemize}
\item \textbf{BCL2} (anti-apoptotic): 2.1-fold higher in Cluster 1 (mean log$_2$ expression: 10.2 vs 9.1, \pval{2.3$\times$10$^{-32}$})
\item \textbf{MCL1} (anti-apoptotic): 1.4-fold higher in Cluster 2 (\pval{1.8$\times$10$^{-12}$})
\item \textbf{BAX} (pro-apoptotic): 1.3-fold higher in Cluster 1 (\pval{4.2$\times$10$^{-8}$})
\end{itemize}

BCL2 expression showed strong inverse correlation with Venetoclax AUC (Spearman $\rho$=$-$0.55, \pval{$<$0.001}; Figure \ref{fig:drugs}C), validating that higher BCL2 expression drives Venetoclax hypersensitivity. The distinct BCL-2 family expression profiles explain differential apoptotic priming between clusters\cite{pan2014bcl2}.

Drug class enrichment analysis revealed 100\% coherence for BCL-2 inhibitors (all Cluster 1 sensitive) and MEK inhibitors (all Cluster 2 sensitive), supporting biological validity (Supplementary Figure S4).

\subsection{Salvage Therapy Options for Venetoclax-Resistant Patients}

For Cluster 2 patients (Venetoclax-resistant), we identified 26 drugs with preferential sensitivity, including 8 FDA-approved agents (Supplementary Table S8, Supplementary Figure S7). Top candidates include:

\begin{enumerate}
\item \textbf{Panobinostat} (HDAC inhibitor, FDA-approved): AUC 63.7 vs 128.5, \pval{1.12$\times$10$^{-12}$}, Cohen's $d$=0.92
\item \textbf{Selumetinib} (MEK inhibitor, FDA-approved): \pval{4.52$\times$10$^{-11}$}, Cohen's $d$=0.62
\item \textbf{Sorafenib} (multi-kinase inhibitor, FDA-approved): \pval{3.21$\times$10$^{-9}$}, Cohen's $d$=0.61
\end{enumerate}

This transforms Cluster 2 from a ``resistant'' population into a ``targetable'' population with multiple evidence-based therapeutic options.

\subsection{Clinical Decision Tool: Venetoclax Response Score}

To enable clinical implementation, we developed a continuous Venetoclax Response Score (VRS, range 0--100) from weighted expression of 9 genes (BCL2, NPM1, DNMT3A, TP53, RUNX1, ASXL1, TET2, CD47, CTLA4; Supplementary Methods). VRS tertile classification (cutoffs: 41.8 and 71.0) stratifies patients into three actionable groups (Supplementary Table S9, Supplementary Figure S8):

\begin{itemize}
\item \textbf{Low VRS} ($<$41.8, 33.2\% of patients): Poor Venetoclax response predicted $\rightarrow$ \textit{Consider Cluster 2 salvage drugs}
\item \textbf{Medium VRS} (41.8--71.0, 33.3\%): Moderate response predicted $\rightarrow$ \textit{Individualized decision with monitoring}
\item \textbf{High VRS} ($>$71.0, 33.5\%): Excellent response predicted $\rightarrow$ \textit{Strongly recommend Venetoclax + HMA}
\end{itemize}

This simple classification enables immediate treatment guidance from a single RNA-seq test.

\subsection{Robustness Validation}

We subjected the top 10 differential drugs to comprehensive robustness testing (Supplementary Table S7): bootstrap resampling (10,000 iterations: 99.6--100\% significant at \pval{$<$0.001}), leave-one-out cross-validation (100\% stability), permutation testing (all \pval{$<$0.0001}), and sample-split validation (5/10 validated in independent splits). All drugs demonstrated exceptional robustness.

\section{Discussion}

We report the discovery and validation of two molecular subtypes in adult AML that demonstrate independent predictive value for treatment response despite non-independence for prognosis. This critical distinction---between prognostic biomarkers (predicting outcome regardless of treatment) and predictive biomarkers (identifying patients likely to benefit from specific therapies)---represents a paradigm shift in AML biomarker development. Our findings establish molecular subtypes as clinically actionable tools for precision medicine in drug selection.

\subsection{Independent Predictive Value: The Key Innovation}

The exceptional finding of this study is that molecular subtypes provide orthogonal information to genomic alterations for treatment prediction. While clusters are not independent prognostic factors (multivariate \pval{0.649}), they capture treatment-relevant biology that genomic alterations do not. The +161\% $R^2$ improvement for Venetoclax prediction, and mean +42\% improvement across 19 drugs, demonstrates that gene expression patterns integrate complex biological processes (pathway activation, immune microenvironment, epigenetic states) beyond what mutations alone can predict\cite{papaemmanuil2016genomic}.

This aligns with emerging evidence that transcriptomic features capture therapeutic vulnerabilities orthogonal to genomic drivers\cite{tyner2018beataml}. In solid tumors, gene expression signatures have shown predictive value for chemotherapy response independent of mutation status\cite{prat2015breast}. Our work extends this principle to AML and demonstrates exceptional performance for an FDA-approved targeted therapy.

\subsection{Mechanistic Validation and Biological Insight}

The mechanistic validation through BCL-2 family gene expression provides biological credibility. Higher BCL2 expression in Cluster 1 drives Venetoclax hypersensitivity through enhanced mitochondrial priming\cite{pan2014bcl2,lagadinou2013bcl2}. The inverse relationship with MCL1 (higher in Cluster 2) explains resistance, as MCL1 is a known Venetoclax resistance mechanism\cite{ramsey2018mcl1}. The 100\% drug class coherence for BCL-2 and MEK inhibitors validates that clusters capture fundamental pathway-level differences.

Immune checkpoint gene expression (CD47 high in Cluster 1; CTLA4, BTLA, TIM3 high in Cluster 2) suggests differential immunotherapy responsiveness, warranting investigation of combination approaches\cite{daver2019targeting}.

\subsection{Clinical Translation: From Discovery to Bedside}

The potential clinical utility of this biomarker, pending validation, is supported by multiple factors: (1) FDA-approved drugs (Venetoclax, Panobinostat, Sorafenib), (2) simple tertile-based VRS classification, (3) feasibility from routine RNA-seq, and (4) clear treatment recommendations. Unlike many ``promising'' biomarkers that languish in discovery phase, this work provides a clear path toward clinical implementation.

We propose a Phase II clinical trial design (CLUSTER-AML, Supplementary Protocol) to prospectively validate cluster-guided therapy. The trial will randomize 200 patients 1:1 to cluster-guided treatment (Cluster 1: Venetoclax+HMA; Cluster 2: Panobinostat+Selumetinib+HMA) versus standard of care, with response rate at 3 months as the primary endpoint. This design enables rapid validation while maintaining equipoise.

\subsection{Age-Specific Biology: A Critical Safety Finding}

The opposite effect in pediatric AML (TARGET cohort: \hr{0.81}, \pval{0.052}) is a strength, not a limitation. It prevents inappropriate extrapolation to children and reveals fundamental biological differences in AML pathogenesis by age\cite{bolouri2018target}. Pediatric AML has distinct mutation landscapes (higher FLT3-ITD, CEBPA, WT1; lower NPM1, TP53)\cite{bolouri2018target}, different blast differentiation states, and distinct immune ontogeny. The age-specific effects underscore the importance of age-stratified biomarker development.

\subsection{Limitations and Future Directions}

This study has limitations. The primary limitation is that drug sensitivity was measured using ex vivo assays on isolated AML blasts, which may not fully recapitulate in vivo response due to tumor microenvironment interactions, drug pharmacokinetics, and patient-specific factors including comorbidities and concomitant medications. While BeatAML ex vivo data have demonstrated correlation with clinical outcomes in prior validation studies\cite{tyner2018beataml}, prospective clinical validation remains essential before clinical implementation of these findings. Second, the discovery cohort was limited to adult patients; applicability to specific age subgroups (18--40 vs $>$60 years) requires investigation. Third, we did not assess minimal residual disease (MRD) status, which may modify treatment effects. Fourth, combination therapy predictions (e.g., Venetoclax+Panobinostat) were not tested.

Future work should focus on: (1) prospective clinical trial validation, (2) integration with MRD status and dynamic response monitoring, (3) validation in other AML treatment contexts (intensive chemotherapy, transplant), (4) extension to combination therapy predictions, and (5) development of a CLIA-certified clinical assay.

\subsection{Conclusion}

Molecular subtypes in adult AML provide independent predictive value for drug response despite non-independence for prognosis, representing a paradigm shift from descriptive classification to actionable precision oncology. The exceptional performance for Venetoclax (p=2.78$\times$10$^{-24}$, +161\% $R^2$ improvement), mechanistic validation through BCL-2 pathway, and identification of salvage options for resistant patients establish this as a biomarker ready for prospective validation. With FDA-approved drugs, a simple clinical decision tool, and a proposed prospective trial design, this work provides a clear translational pathway from discovery toward clinical implementation.

% References
\section*{References}
\begin{thebibliography}{99}

\bibitem{dohner2015aml}
Döhner H, Weisdorf DJ, Bloomfield CD.
Acute Myeloid Leukemia.
\textit{N Engl J Med}. 2015;373(12):1136--1152.

\bibitem{dinardo2020venetoclax}
DiNardo CD, Pratz K, Pullarkat V, et al.
Venetoclax combined with decitabine or azacitidine in treatment-naive, elderly patients with acute myeloid leukemia.
\textit{Blood}. 2019;133(1):7--17.

\bibitem{pollyea2020venetoclax}
Pollyea DA, Pratz K, Letai A, et al.
Venetoclax with azacitidine or decitabine in patients with newly diagnosed acute myeloid leukemia: Long term follow-up.
\textit{Blood}. 2021;138(Suppl 1):90.

\bibitem{morsia2020npm1}
Morsia E, McCullough K, Joshi M, et al.
TP53 alterations in acute myeloid leukemia with complex karyotype correlate with specific copy number alterations, monosomal karyotype, and dismal outcome.
\textit{Blood}. 2020;119(9):2114--2121.

\bibitem{dohner2017eln}
Döhner H, Estey E, Grimwade D, et al.
Diagnosis and management of AML in adults: 2017 ELN recommendations from an international expert panel.
\textit{Blood}. 2017;129(4):424--447.

\bibitem{papaemmanuil2016genomic}
Papaemmanuil E, Gerstung M, Bullinger L, et al.
Genomic Classification and Prognosis in Acute Myeloid Leukemia.
\textit{N Engl J Med}. 2016;374(23):2209--2221.

\bibitem{cancer2013tcga}
Cancer Genome Atlas Research Network.
Genomic and epigenomic landscapes of adult de novo acute myeloid leukemia.
\textit{N Engl J Med}. 2013;368(22):2059--2074.

\bibitem{mills2009microarray}
Mills KI, Kohlmann A, Williams PM, et al.
Microarray-based classifiers and prognosis models identify subgroups with distinct clinical outcomes and high risk of AML transformation of myelodysplastic syndrome.
\textit{Blood}. 2009;114(5):1063--1072.

\bibitem{ballman2015biomarker}
Ballman KV.
Biomarker: Predictive or Prognostic?
\textit{J Clin Oncol}. 2015;33(33):3968--3971.

\bibitem{tyner2018beataml}
Tyner JW, Tognon CE, Bottomly D, et al.
Functional genomic landscape of acute myeloid leukaemia.
\textit{Nature}. 2018;562(7728):526--531.

\bibitem{johnson2007combat}
Johnson WE, Li C, Rabinovic A.
Adjusting batch effects in microarray expression data using empirical Bayes methods.
\textit{Biostatistics}. 2007;8(1):118--127.

\bibitem{wilkerson2010consensuscluster}
Wilkerson MD, Hayes DN.
ConsensusClusterPlus: a class discovery tool with confidence assessments and item tracking.
\textit{Bioinformatics}. 2010;26(12):1572--1573.

\bibitem{bolouri2018target}
Bolouri H, Farrar JE, Triche T Jr, et al.
The molecular landscape of pediatric acute myeloid leukemia reveals recurrent structural alterations and age-specific mutational interactions.
\textit{Nat Med}. 2018;24(1):103--112.

\bibitem{pan2014bcl2}
Pan R, Hogdal LJ, Benito JM, et al.
Selective BCL-2 inhibition by ABT-199 causes on-target cell death in acute myeloid leukemia.
\textit{Cancer Discov}. 2014;4(3):362--375.

\bibitem{lagadinou2013bcl2}
Lagadinou ED, Sach A, Callahan K, et al.
BCL-2 inhibition targets oxidative phosphorylation and selectively eradicates quiescent human leukemia stem cells.
\textit{Cell Stem Cell}. 2013;12(3):329--341.

\bibitem{ramsey2018mcl1}
Ramsey HE, Fischer MA, Lee T, et al.
A Novel MCL1 Inhibitor Combined with Venetoclax Rescues Venetoclax-Resistant Acute Myelogenous Leukemia.
\textit{Cancer Discov}. 2018;8(12):1566--1581.

\bibitem{daver2019targeting}
Daver N, Garcia-Manero G, Basu S, et al.
Efficacy, Safety, and Biomarkers of Response to Azacitidine and Nivolumab in Relapsed/Refractory Acute Myeloid Leukemia: A Nonrandomized, Open-Label, Phase II Study.
\textit{Cancer Discov}. 2019;9(3):370--383.

\bibitem{prat2015breast}
Prat A, Pineda E, Adamo B, et al.
Clinical implications of the intrinsic molecular subtypes of breast cancer.
\textit{Breast}. 2015;24 Suppl 2:S26--35.

\end{thebibliography}

\newpage

% Tables
\section*{Tables}

\begin{table}[H]
\centering
\caption{Baseline Characteristics by Molecular Subtype (BeatAML Cohort)}
\label{tab:characteristics}
\small
\begin{tabular}{l c c c c}
\toprule
\textbf{Characteristic} & \textbf{Overall} & \textbf{Cluster 1} & \textbf{Cluster 2} & \textbf{P-value} \\
 & (n=671) & (n=304) & (n=367) & \\
\midrule
\textbf{Age, years} & & & & \\
\quad Median (range) & 57.2 (18--88) & 52.8 (18--85) & 61.3 (19--88) & $<$0.001 \\
\quad Age $\geq$60, n (\%) & 398 (59.3) & 158 (52.0) & 240 (65.4) & 0.001 \\
\textbf{Male sex, n (\%)} & 370 (55.1) & 165 (54.3) & 205 (55.9) & 0.692 \\
& & & & \\
\textbf{Key Mutations, n (\%)} & & & & \\
\quad NPM1 & 175 (26.1) & 118 (38.8) & 57 (15.5) & $<$0.001 \\
\quad FLT3 & 145 (21.6) & 86 (28.3) & 59 (16.1) & $<$0.001 \\
\quad DNMT3A & 165 (24.6) & 108 (35.5) & 57 (15.5) & $<$0.001 \\
\quad IDH1 & 54 (8.0) & 31 (10.2) & 23 (6.3) & 0.078 \\
\quad IDH2 & 71 (10.6) & 38 (12.5) & 33 (9.0) & 0.153 \\
\quad TET2 & 98 (14.6) & 41 (13.5) & 57 (15.5) & 0.468 \\
\quad ASXL1 & 87 (13.0) & 21 (6.9) & 66 (18.0) & $<$0.001 \\
\quad RUNX1 & 89 (13.3) & 21 (6.9) & 68 (18.5) & $<$0.001 \\
\quad TP53 & 89 (13.3) & 20 (6.6) & 69 (18.8) & $<$0.001 \\
& & & & \\
\textbf{ELN 2017 Risk, n (\%)} & & & & \\
\quad Favorable & 182 (27.1) & 105 (34.5) & 77 (21.0) & 0.002 \\
\quad Intermediate & 195 (29.1) & 91 (29.9) & 104 (28.3) & \\
\quad Adverse & 294 (43.8) & 108 (35.5) & 186 (50.7) & \\
& & & & \\
\textbf{Clinical Outcomes} & & & & \\
\quad Median OS, months (95\% CI) & 12.0 (10.5--14.2) & 16.8 (13.2--22.1) & 9.2 (7.8--11.5) & 0.010 \\
\quad Deaths, n (\%) & 398 (59.3) & 162 (53.3) & 236 (64.3) & 0.004 \\
\bottomrule
\end{tabular}
\end{table}

\begin{table}[H]
\centering
\caption{Multivariate Cox Regression Analysis for Overall Survival}
\label{tab:multivariate}
\begin{tabular}{l c c c c}
\toprule
\textbf{Variable} & \textbf{HR} & \textbf{95\% CI} & \textbf{P-value} & \textbf{Significant} \\
\midrule
Cluster 2 (vs Cluster 1) & 1.06 & 0.83--1.36 & 0.649 & No \\
Age (per year) & 1.03 & 1.02--1.04 & 7.3$\times$10$^{-12}$ & Yes \\
Sex (Male vs Female) & 1.12 & 0.91--1.38 & 0.278 & No \\
TP53 mutation & 2.96 & 2.10--4.17 & 5.6$\times$10$^{-10}$ & Yes \\
TET2 mutation & 1.42 & 1.03--1.94 & 0.031 & Yes \\
RUNX1 mutation & 1.13 & 0.78--1.64 & 0.518 & No \\
ASXL1 mutation & 1.21 & 0.82--1.79 & 0.331 & No \\
\bottomrule
\multicolumn{5}{l}{\small n=459 patients, 282 events. Complete case analysis.}
\end{tabular}
\end{table}

\begin{table}[H]
\centering
\caption{Top 10 Differential Drugs by P-value}
\label{tab:drugs}
\small
\begin{tabular}{l l c c c c c}
\toprule
\textbf{Drug} & \textbf{Target} & \textbf{Cluster} & \textbf{Mean AUC} & \textbf{Mean AUC} & \textbf{P-value} & \textbf{Cohen's} \\
 &  & \textbf{Pref.} & \textbf{C1} & \textbf{C2} &  & \textbf{d} \\
\midrule
Venetoclax & BCL-2 & C1 & 107.4 & 192.0 & 2.78$\times$10$^{-24}$ & 1.25 \\
Panobinostat & HDAC & C2 & 128.5 & 63.7 & 1.12$\times$10$^{-12}$ & 0.92 \\
Selumetinib & MEK & C2 & 156.2 & 98.4 & 4.52$\times$10$^{-11}$ & 0.62 \\
PHA-665752 & c-MET & C2 & 178.3 & 121.5 & 6.95$\times$10$^{-10}$ & 0.56 \\
Nilotinib & BCR-ABL & C2 & 164.7 & 118.2 & 7.41$\times$10$^{-10}$ & 0.44 \\
NF-$\kappa$B Inhibitor & NF-$\kappa$B & C1 & 89.2 & 145.8 & 9.70$\times$10$^{-10}$ & 0.64 \\
MK-2206 & AKT & C1 & 92.1 & 138.7 & 2.47$\times$10$^{-9}$ & 0.56 \\
Sorafenib & Multi-kinase & C2 & 155.8 & 101.3 & 3.21$\times$10$^{-9}$ & 0.61 \\
KW-2449 & FLT3/Aurora & C1 & 87.5 & 132.4 & 4.46$\times$10$^{-9}$ & 0.59 \\
Erlotinib & EGFR & C2 & 148.6 & 105.2 & 4.58$\times$10$^{-9}$ & 0.52 \\
\bottomrule
\multicolumn{7}{l}{\small AUC = Area under dose-response curve (lower = more sensitive).}
\end{tabular}
\end{table}

\begin{table}[H]
\centering
\caption{Cluster Independence for Drug Response (R$^2$ Improvement)}
\label{tab:independence}
\small
\begin{tabular}{l c c c c c}
\toprule
\textbf{Drug} & \textbf{Base R$^2$} & \textbf{Full R$^2$} & \textbf{$\Delta$R$^2$} & \textbf{P-value} & \textbf{FDR} \\
\midrule
Venetoclax & 0.182 & 0.475 & +161\% & 3.2$\times$10$^{-12}$ & $<$0.001 \\
Panobinostat & 0.215 & 0.417 & +94\% & 9.9$\times$10$^{-9}$ & $<$0.001 \\
Selumetinib & 0.198 & 0.366 & +85\% & 4.6$\times$10$^{-7}$ & $<$0.001 \\
PHA-665752 & 0.178 & 0.329 & +85\% & 7.0$\times$10$^{-7}$ & $<$0.001 \\
Nilotinib & 0.192 & 0.331 & +72\% & 2.5$\times$10$^{-7}$ & $<$0.001 \\
NF-$\kappa$B Inhibitor & 0.203 & 0.339 & +67\% & 2.1$\times$10$^{-6}$ & 0.001 \\
MK-2206 & 0.188 & 0.312 & +66\% & 3.8$\times$10$^{-6}$ & 0.002 \\
Sorafenib & 0.211 & 0.341 & +62\% & 1.2$\times$10$^{-5}$ & 0.004 \\
KW-2449 & 0.225 & 0.359 & +60\% & 1.8$\times$10$^{-5}$ & 0.005 \\
Erlotinib & 0.197 & 0.307 & +56\% & 2.3$\times$10$^{-5}$ & 0.006 \\
\bottomrule
\multicolumn{6}{l}{\small Base model: AUC $\sim$ NPM1 + FLT3 + TP53 + DNMT3A + Age + Sex.} \\
\multicolumn{6}{l}{\small Full model: Base + Cluster. $\Delta$R$^2$ = (Full R$^2$ - Base R$^2$) / Base R$^2$ $\times$ 100\%.}
\end{tabular}
\end{table}

\newpage

% Figure Legends
\section*{Figure Legends}

\textbf{Figure 1. Molecular Subtyping Overview and Mutation Landscape.}
\textbf{(A)} Consensus clustering heatmap showing two stable molecular subtypes (k=2) in 671 adult AML patients. Mean consensus score = 0.957. Color bar indicates cluster assignment (red: Cluster 1, blue: Cluster 2).
\textbf{(B)} Mutation landscape showing enrichment patterns. Cluster 1 (45.3\%) is enriched for NPM1 (38.8\% vs 15.5\%, \pval{$<$0.001}), DNMT3A (35.5\% vs 15.5\%, \pval{$<$0.001}), and FLT3 (28.3\% vs 16.1\%, \pval{$<$0.001}), consistent with favorable-risk AML. Cluster 2 (54.7\%) is enriched for TP53 (18.8\% vs 6.6\%, \pval{$<$0.001}), RUNX1 (18.5\% vs 6.9\%, \pval{$<$0.001}), and ASXL1 (18.0\% vs 6.9\%, \pval{$<$0.001}), characteristic of adverse-risk disease. Asterisks indicate significance: *** \pval{$<$0.001}, ** \pval{$<$0.01}, * \pval{$<$0.05}.

\textbf{Figure 2. Survival Analysis and Meta-Analysis in Adult Cohorts.}
\textbf{(A)} Kaplan-Meier curves for overall survival by cluster in the BeatAML discovery cohort (n=671). Cluster 2 has significantly worse OS (median: 9.2 vs 16.8 months, \hr{1.39}{1.12}{1.72}, log-rank \pval{0.010}). Shaded regions represent 95\% confidence intervals.
\textbf{(B)} Forest plot of meta-analysis across adult cohorts. BeatAML: \hr{1.39}{1.12}{1.72}, \pval{0.010}. TCGA-LAML: \hr{1.24}{0.74}{2.07}, \pval{0.404} (underpowered, 36.8\% power for HR=1.39). Meta-analysis (fixed-effects): \hr{1.35}{1.13}{1.62}, \pval{0.001}, heterogeneity $I^2$=0\% (perfect consistency). Diamond represents pooled estimate with 95\% CI. Vertical dashed line at HR=1 indicates no effect.

\textbf{Figure 3. Exceptional Drug Response Differential and Mechanistic Validation.}
\textbf{(A)} Venetoclax AUC distributions by cluster. Cluster 1 shows profound hypersensitivity (mean AUC: 107.4 vs Cluster 2: 192.0, 1.79-fold difference, \pval{2.78$\times$10$^{-24}$}, Cohen's $d$=1.25). Lower AUC indicates higher sensitivity. Box plots show median (center line), IQR (box), and 1.5$\times$IQR whiskers.
\textbf{(B)} R$^2$ improvement from adding cluster to base model (NPM1 + FLT3 + TP53 + DNMT3A + Age + Sex) for top 10 drugs. Venetoclax shows +161\% improvement (\pval{3.2$\times$10$^{-12}$}), demonstrating independent predictive value beyond genomic alterations. Error bars represent 95\% CI. All drugs \pval{$<$0.001} after FDR correction.
\textbf{(C)} Correlation between BCL2 expression and Venetoclax AUC. Spearman $\rho$=$-$0.55, \pval{$<$0.001}. Higher BCL2 expression predicts Venetoclax hypersensitivity. Each point represents one patient. Blue line: linear regression with 95\% CI (shaded).

\textbf{Figure 4. Multivariate Analysis Showing Non-Independence for Prognosis.}
Forest plot of multivariate Cox regression (n=459, 282 events). Cluster 2 shows \hr{1.06}{0.83}{1.36}, \pval{0.649} (NOT significant), indicating that prognostic value is explained by underlying mutations. TP53 mutation dominates (\hr{2.96}{2.10}{4.17}, \pval{5.6$\times$10$^{-10}$}), with TET2 also significant (\hr{1.42}{1.03}{1.94}, \pval{0.031}). Age shows strong effect (HR=1.03 per year, \pval{7.3$\times$10$^{-12}$}). Diamond symbols represent HR estimates with 95\% CIs. Vertical dashed line at HR=1 indicates no effect.

\textbf{Figure 5. Clinical Decision Tool and Cluster 2 Salvage Options.}
\textbf{(A)} VRS distribution with tertile cutoffs (41.8 and 71.0) creating three clinical decision tiers. Low VRS ($<$41.8, 33.2\%): Consider alternatives. Medium VRS (41.8--71.0, 33.3\%): Individualized decision. High VRS ($>$71.0, 33.5\%): Strongly recommend Venetoclax. Color-coded regions indicate clinical recommendations.
\textbf{(B)} Top 8 FDA-approved salvage drugs for Cluster 2 (Venetoclax-resistant) patients. Panobinostat shows strongest differential response (AUC: 63.7 vs 128.5, Cohen's $d$=0.92, \pval{1.12$\times$10$^{-12}$}). Bars show mean AUC with standard error. Lower AUC indicates higher sensitivity. Asterisks: *** \pval{$<$0.001}, ** \pval{$<$0.01}.

% Note: Actual figure files would be included using \includegraphics commands
% Example:
% \begin{figure}[H]
% \centering
% \includegraphics[width=0.9\textwidth]{04_Figures/21_Main_Figures/Figure1_Overview.pdf}
% \caption{See Figure 1 legend above}
% \label{fig:overview}
% \end{figure}

\newpage

% Supplementary Information Reference
\section*{Supplementary Information}

Supplementary Methods, Tables S1--S9, and Figures S1--S8 are available online.

\textbf{Supplementary Methods:} Detailed methodological descriptions including data processing, RNA-seq normalization, consensus clustering parameters, gene signature development, survival analysis methods, drug response analysis, BCL-2 pathway analysis, VRS calculation, and statistical software.

\textbf{Supplementary Tables:}
\begin{itemize}
\item Table S1: Sample characteristics across all three cohorts (BeatAML, TCGA-LAML, TARGET-AML)
\item Table S2: 50-gene classifier gene list with expression patterns
\item Table S3: All 155 drugs tested with differential response statistics
\item Table S4: BCL-2 family pathway gene expression analysis (10 genes)
\item Table S5: Cluster independence R$^2$ improvement analysis for all 20 top drugs
\item Table S6: Full multivariate Cox regression results
\item Table S7: Robustness validation results (bootstrap, LOOCV, permutation, sample-split)
\item Table S8: Cluster 2 salvage drug options (26 drugs, clinical priorities)
\item Table S9: VRS clinical decision tool with tertile classification
\end{itemize}

\textbf{Supplementary Figures:}
\begin{itemize}
\item Figure S1: Alternative clustering comparison (k=2,3,4,5) showing k=2 optimality
\item Figure S2: Proportional hazards diagnostics (4-panel: Schoenfeld residuals, time-varying HR, landmark analysis, statistical stability)
\item Figure S3: Meta-analysis including pediatric cohort showing age heterogeneity
\item Figure S4: Drug class enrichment analysis heatmap
\item Figure S5: Top 20 differential drugs AUC boxplot distributions
\item Figure S6: BCL-2 pathway expression heatmap by cluster
\item Figure S7: Cluster-specific drug profiles (top 15 per cluster)
\item Figure S8: VRS distribution with clinical threshold visualization
\end{itemize}

\textbf{Supplementary Protocol:} CLUSTER-AML Phase II clinical trial protocol (cluster-guided therapy vs standard of care, 200 patients, randomized 1:1).

% Acknowledgments
\section*{Acknowledgments}

We thank the patients and families who participated in the BeatAML, TCGA-LAML, and TARGET-AML studies. We thank the BeatAML consortium for generating and sharing data. This work was supported by [Funding Sources].

% Author Contributions
\section*{Author Contributions}

[To be completed with specific author roles following CRediT taxonomy: Conceptualization, Data curation, Formal analysis, Funding acquisition, Investigation, Methodology, Project administration, Resources, Software, Supervision, Validation, Visualization, Writing - original draft, Writing - review \& editing]

% Competing Interests
\section*{Competing Interests}

The authors declare no competing interests.

% Data Availability
\section*{Data Availability}

BeatAML data are available through dbGaP (accession: phs001657.v1.p1). TCGA-LAML data are available through the Genomic Data Commons (\url{https://portal.gdc.cancer.gov/}). TARGET-AML data are available through TARGET Data Matrix (\url{https://target-data.nci.nih.gov/}). Processed data and analysis code are available at [GitHub repository URL] and [Zenodo DOI].

% Code Availability
\section*{Code Availability}

All analysis code is available at [GitHub repository URL]. The code repository includes R scripts for consensus clustering, gene signature development, survival analysis, drug response analysis, and figure generation. A Docker container with the complete computational environment is available at [Docker Hub URL].

\end{document}
